\documentclass[journal,12pt,twocolumn]{IEEEtran}
\usepackage{hyperref}
\usepackage{romannum}
\usepackage{tfrupee}
\usepackage{enumitem}
\usepackage{amsmath}
\usepackage{amssymb}
\usepackage{makecell} 
\title{Assignment 3}
\author{JARPULA BHANU PRASAD - AI21BTECH11015}
\date{April 2022}
\begin{document}
\maketitle
\noindent \Large\underline{Download codes from}:\\
\noindent\large Download python code from - \href{https://github.com/jarpula-Bhanu/Assignment-3/blob/main/codes/fredat.py}{Python}\\ Download latex code from - \href{https://github.com/jarpula-Bhanu/Assignment-3/blob/main/Assignment3.tex}{Latex}

\section{\large\underline{Problem-CBSE-9th Q)example-2}}
\large \noindent Q)Consider the marks obtained(out of 100 marks) by 30 students of Class \Romannum{9} of a school: 
\begin{table}[ht!]
\begin{center}
		\normalsize \begin{tabular}{c c c c c c c c c c}

$10$ & $20$ & $36$ & $92$ & $95$ & $40$ & $50$ & $56$ & $60$ & $70$ \\

$92$ & $88$ & $80$ & $70$ & $72$ & $70$ & $36$ & $40$ & $36$ & $40$ \\

$92$ & $40$ & $50$ & $50$ & $56$ & $60$ & $70$ & $60$ & $60$ & $88$ \\

\end{tabular}
		\vspace*{5pt}
		\caption{}
		\label{table:table1}
\end{center}	
	\end{table}\\
construct the frequency distribution table.
\section{\large\underline{Solution}}
\noindent The number of students who have obtained a certain number of marks is called the $frequency$ of those marks. For instance, 4 students got 70 marks. So the frequency of 70 marks is 4. To make the data more easily understandable, we write it in a table, as given:
\begin{table}[ht!]
\begin{center}
		\normalsize \begin{tabular}{|c|c|}

\hline
\textbf{Marks} & \textbf{\thead{Number of students \\ (i.e.,the frequency)}} \\
\hline
$10$   & $1$ \\
$20$   & $1$ \\
$36$   & $3$ \\
$40$   & $4$ \\
$50$   & $3$ \\
$56$   & $2$ \\
$60$   & $4$ \\
$70$   & $4$ \\
$72$   & $1$ \\
$80$   & $1$ \\
$88$   & $2$ \\
$92$   & $3$ \\
$95$   & $1$ \\
\hline
\textbf{Total} & $30$\\
\hline

\end{tabular}
		\vspace*{5pt}
		\caption{}
		\label{table:table2}	
\end{center}
\end{table}\\

\noindent Table \ref{table:table2} is called an ungrouped frequency distribution table or simply a frequency distribution table. Note that you can use also tally marks in preparing these tables.

\end{document}
